\section{NAO}
\subsection{NAO-kode}
NAO snakker og går
\lstinputlisting[language=Python,firstline=2]{../NAOProsjekt/NAO20.py}
NAO lokaliserer en rød ball og går til den
\lstinputlisting[language=Python,firstline=14]{../NAOProsjekt/NAOtesting.py}

\section{ABB}
\subsection{Utstyrsliste}
\begin{table}[!htb]
    \centering
    % \textbf{Programvare}\par\medskip 
    \caption{Oversikt over hvilken programvare som ble benyttet under dette prosjektet}
    \begin{tabular}{|l|l|}
        \hline
        \textbf{Program} & \textbf{Type}\\
        \hline
        RobotStudio (RAPID) & Programmeringsmiljø\\
        \hline
        Python & Programmeringsspråk \\
        \hline
        OpenCV & Bibliotek \\
        \hline
        socket & Bibliotek \\
        \hline
    \end{tabular} \newline
    \label{app:soft}
\end{table}
\begin{table}[!htb]
    \centering
    % \textbf{Maskinvare}\par\medskip 
    \caption{Oversikt over hvilken maskinvare som ble benyttet under dette prosjektet}
    \begin{tabular}{|l|l|l|}
        \hline
        \textbf{Utstyr} & \textbf{Type} & \textbf{Fabrikant}\\
        \hline
        Robotarm & IRB 140, 6kg, 0.81m & ABB\\
        \hline
        Kamera & UI-3360CP & iDS\\
        \hline
        Datamaskin & Laptop & Windows \\
        \hline
        FlexPendant & Kontroller & ABB \\
        \hline
        Sugekopp & Verktøy til robotarm &  \\
        \hline
    \end{tabular} \newline
    \label{app:hard}
\end{table}

\subsection{Python-kode}
Objektgjenkjenning
\lstinputlisting[language=Python, firstline=3, lastline=158]{../ABB_Prosjekt/center_contour.py}
Kommunikasjon med robot
\lstinputlisting[language=Python, firstline=3]{../ABB_Prosjekt/Client_ABB.py}

\subsection{RAPID-kode (Utdrag)}
Kommunikasjon med klient
\lstinputlisting[style=rapid]{../ABB_Prosjekt/RAPID/Communication.mod}
Flytting av objekter
\lstinputlisting[style=rapid]{../ABB_Prosjekt/RAPID/Movement.mod}
Deklarering av variable- og konstante verdier, samt kjøring av hovedprosedyre
\lstinputlisting[style=rapid]{../ABB_Prosjekt/RAPID/MainModule.mod}